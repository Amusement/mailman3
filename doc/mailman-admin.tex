\documentclass{howto}

\title{GNU Mailman - List Administration Manual}

% Increment the release number whenever significant changes are made.
% The author and/or editor can define 'significant' however they like.
\release{1.0}

% At minimum, give your name and an email address.  You can include a
% snail-mail address if you like.
\author{Barry A. Warsaw}
%\authoraddress{barry@zope.com}

\date{\today}			% XXX update before tagging release!
\release{2.1}			% software release, not documentation
\setreleaseinfo{b4}		% empty for final release
\setshortversion{2.1}		% major.minor only for software

\begin{document}
\maketitle

% This makes the Abstract go on a separate page in the HTML version;
% if a copyright notice is used, it should go immediately after this.
%
\ifhtml
\chapter*{Front Matter\label{front}}
\fi

% Copyright statement should go here, if needed.
% ...

% The abstract should be a paragraph or two long, and describe the
% scope of the document.
\begin{abstract}
\noindent
This document describes the list administrator's interface for GNU
Mailman 2.1.  It contains information a list owner would need to
configure their list, either through the web interface or through
email.  It also covers the moderator's interface for approving held
messages and subscription notices, and the web interface for creating
new mailing lists.  In general, it does not cover the command line
interface to Mailman, installing Mailman, or interacting with Mailman
from the point of view of the user.  That information is covered in
other manuals.
\end{abstract}

\tableofcontents

\section{Introduction to GNU Mailman}

GNU Mailman is software that lets you manage electronic mailing lists.
It supports a wide range of mailing list types, such as general
discussion lists and announce-only lists.  Mailman has extensive
features for controlling the privacy of your lists, distributing your
list as personalized postings or digests, gatewaying postings to and
from Usenet, and providing automatic bounce detection.  Mailman
provides a built-in archiver, multiple natural languages, as well as
advanced content and topic filtering.

Mailman provides several interfaces to its functionality.  Most list
administrators will primarily use the web interface to customize their
lists.  There is also a limited email command interface to the
administrative functions, as well as a command line interface if you
have shell access on the Mailman server.  This document does not cover
the command line interface; see the GNU Mailman site administrator's
manual for more details.

\subsection{A List's Email Addresses}

Every mailing list has a set of email addresses that messages can be
sent to.  There's always one address for posting messages to the list,
one address that bounces will be sent to, and addresses for processing
email commands.  For example, for a mailing list called
\var{mylist@example.com}, you'd find these addresses:

\begin{itemize}
\item mylist@example.com -- this is the email address people should
      use for new postings to the list.

\item mylist-join@example.com -- by sending a message to this address,
      a new member can request subscription to the list.  Both the
      \mailheader{Subject} header and body of such a message are
      ignored.  Note that mylist-subscribe@example.com is an alias for
      the -join address.

\item mylist-leave@example.com -- by sending a message to this address,
      a member can request unsubscription from the list.  As with the
      -join address, the \mailheader{Subject} header and body of the
      message is ignored.  Note that mylist-unsubscribe@example.com is
      an alias for the -leave address.

\item mylist-owner@example.com -- This address reaches the list owner
      and list moderators directly.

\item mylist-request@example.com -- This address reaches a mail robot
      which processes email commands that can be used to set member
      subscription options, as well as process other commands.

\item mylist-bounces@example.com -- This address receives bounces from
      members who's addresses have become either temporarily or
      permanently inactive.  The -bounces address is also a mail robot
      that processes bounces and automatically disables or removes
      members as configured in the bounce processing settings.  Any
      bounce messages that are either unrecognized, or do not seem to
      contain member addresses, are forwarded to the list
      administrators.

\item mylist-confirm@example.com -- This address is another email
      robot, which processes confirmation messages for subscription
      and unsubscription requests.
\end{itemize}

There's also an -admin address which also reaches the list
administrators, but this address only exists for compatibility with
older versions of Mailman.

\subsection{Administrative Roles}

There are two primary administrative roles for each mailing list, a
list owner and a list moderator.  A list owner is allowed to change
various settings of the list, such as the privacy and archiving
policies, the content filtering settings, etc.  The list owner is also
allowed to subscribe or invite members, unsubscribe members, and
change any member's subscription options.

The list moderator on the other hand, is only allowed to approve or
reject postings and subscription requests.  The list moderator can
also do things like clear a member's moderation flag, or add an
address to a list of approved non-member posters.

Normally, the list owner and list moderator are the same person.  In
fact, the list owner can always do all the tasks a list moderator can
do.  Access to both the owner's configuration pages, and the
moderation pages are protected by the same password.  However, if the
list owner wants to delegate posting and subscription approval
authority to other people, a separate list moderator password can be
set, giving moderators access to the approval pages, but not the
configuration pages.  In this setup, list owners can still moderate
the list, of course.

In the sections that follow, we'll often use the terms list owner and
list administrator interchangably, meaning both roles.  When
necessary, we'll distinguish the list moderator explicitly.

\subsection{A List's Web Pages}

Every mailing list is also accessible by a number of web pages.  Note
that the exact urls is configurable by the site administrator, so they
may be different than what's described below.  We'll describe the most
common default configuration, but check with your site administrator
or hosting service for details.

Mailman provides a set of web pages that list members use to get
information about the list, or manage their membership options.  There
are also list archive pages, for browsing an online web-based archive
of the list traffic.  These are described in more detail in the GNU
Mailman user's manual.

Mailman also provides a set of pages for configuring an individual
list, as well as a set of pages for disposing of posting and
subscription requests.

For a mailing list called \var{mylist} hosted at the domain
\var{lists.example.com}, you would typically access the administrative
pages by going to \code{http://lists.example.com/mailman/admin/mylist}.
The first time you visit this page, you will be presented with a login
page, asking for the list owner's password.  When you enter the
password, Mailman will store a session cookie in your browser, so you
don't have to re-authenticate for every action you want to take.  This
cookie is stored only until you exit your browser.

To access the administrative requests page, you'd visit
\code{http://lists.example.com/mailman/admindb/mylist} (note the
\emph{admindb} url as opposed to the \emph{admin} url).  Again, the
first time you visit this page, you'll be presented with a login page,
on which you can enter either the list moderator password or the list
owner password.  Again, a session cookie is dropped in your browser.
Note also that if you've previously logged in as the list owner, you
do not need to re-login to access the administrative requests page.

\subsection{Basic Architectural Overview}

This section will outline the basic architecture of GNU Mailman, such
as how messages are processed by the sytem.  Without going into lots
of detail, this information will help you understand how the
configuration options control Mailman's functionality.

When mail enters the system from your mail server, it is dropped into
one of several Mailman \emph{queues} depending on the address the
message was sent to.  For example, if your system has a mailing list
named \var{mylist} and your domain is \var{example.com}, people can
post messages to your list by sending them to
\var{mylist@example.com}.  These messages will be dropped into the
\emph{incoming} queue, which is also colloquially called the
\emph{moderate-and-munge} queue.  The incoming queue is where most of
the approval process occurs, and it's also where the message is
prepared for sending out to the list membership.

There are separate queues for the built-in archiver, the bounce
processor, the email command processor, as well as the outgoing email
and news queues.  There's also a queue for messages generated by the
Mailman system.  Each of these queues typically has one \emph{queue
runner} (or ``qrunner'') that processes messages in the queue.  The
qrunners are idle when there are no messages to process.

Every message in the queues are represented by two files, a message
file and a metadata file.  Both of these files share the same base
name, which is a combination of a unique hash and the Unix time that
the message was received.  The metadata file has a suffix of
\file{.db} and the message file has a suffix of either \file{.msg} if
stored in plain text, or \file{.pck} if stored in a more efficient
internal representation\footnote{Specifically, a Python pickle}.

As a message moves through the incoming queue, it performs various
checks on the message, such as whether it matches one of the
moderation criteria, or contains disallowed MIME types.  Once a
message is approved for sending to the list membership, the message is
prepared for sending by deleting, adding, or changing message headers,
adding footers, etc.  Messages in the incoming queue may also be
stored for appending to digests.

\section{The List Configuration Pages}

After logging into the list configuration pages, you'll see the
configuration options for the list, grouped in categories.  All the
administrative pages have some common elements.  In the upper section,
you'll see two columns labeled ``Configuration Categories''.  Some
categories have sub-categories which are only visible when you click
on the category link.  The first page you see after logging in will be
the ``General Options'' category.  The specific option settings for
each category are described below.

On the right side of the top section, you'll see a column labeled
``Other Administrative Activities''.  Here you'll find some other
things you can do to your list, as well as convenient links to the
list information page and the list archives.  Note the big ``Logout''
link; use this if you're finished configuring your list and don't want
to leave the session cookie active in your browser.

The other thing you'll see in the upper section is the emergency
moderation checkbox.  This is described in more detail in the
emergency moderation section.

Below this common header, you'll find a list of this category's
configuration variables, arranged in two columns.  In the left column
is a brief description of the option, which also contains a
``details'' link.  For many of the variables, more details are
available describing the semantics of the various available settings,
or information on the interaction between this setting and other list
options.  Clicking on the details link brings up a page which contains
only the information for that option, as well as a button for
submitting your setting, and a link back to the category page.

On the right side of the two-column section, you'll see the variable's
current value.  Some variables may present a limited set of values,
via radio button or check box arrays.  Other variables may present
text entry boxes of one or multiple lines.  Most variables control
settings for the operation of the list, but others perform immediate
actions (these are clearly labeled).

At the bottom of the page, you'll find a ``Submit'' button and a
footer with some more useful links and a few logos.  Hitting the
submit button commits your list settings, after they've been
validated.  Any invalid values will be ignored and an error message
will be displayed at the top of the resulting page.  The results page
will always be the category page that you submitted.

\subsection{The General Options Category}

The General Options category is where you can set a variety of
variables that affect basic behavior and public information.  In the
descriptions that follow, the variable name is given first, along with
an overview and a description of what that variable controls.

\subsubsection{General list personality}

These variables, grouped under the general list personality section,
control some public information about the mailing list.

\begin{description}
\item[real_name] --
    Every mailing list has both a \emph{posting name} and a \emph{real
    name}.  The posting name shows up in urls and in email addresses,
    e.g. the \code{mylist} in \code{mylist@example.com}.  The posting
    name is always presented in lower case, with alphanumeric
    characters and no spaces.  The list's real name is used in some
    public information and email responses, such as in the general
    list overview.  The real name can differ from the posting name by
    case only.  For example, if the posting name is \code{mylist}, the
    real name can be \code{Posting}.

\item[owner] --
    This variable contains a list of email addresses, one address per
    line, of the list owners.  These addresses are used whenever the
    list owners need to be contacted, either by the system or by end
    users.  Often, these addresses are used in combination with the
    \code{moderator} addresses (see below).

\item[moderator] --
    This variable contains a list of email addresses, one address per
    line, of the list moderators.  These addresses are often used in
    combination with the \code{owner} addresses.  For example, when
    you email \code{mylist-owner@example.com}, both the owner and
    moderator addresses will receive a copy of the message.

\item[description] --
    In the general list overview page, which shows you every available
    mailing list, each list is displayed with a short description.
    The contents of this variable is that description.  Note that in
    emails from the mailing list, this description is also used in the
    comment section of the \mailheader{To} address.  This text should
    be relatively short and no longer than one line.

\item[info] --
    This variable contains a longer description of the mailing list.
    It is included at the top of the list's information page, and it
    can contain HTML.  However, blank lines will be automatically
    converted into paragraph breaks.  Preview your HTML though,
    because unclosed or invalid HTML can prevent display of parts of
    the list information page.

\item[subject_prefix] --
    This is a string that will be prepended to the
    \mailheader{Subject} header of any message posted to the list.
    For example, if a message is posted to the list with a
    \mailheader{Subject} like:

    \begin{verbatim}
    Subject: This is a message
    \end{verbatim}

    and the \code{subject_prefix} is \code{[My List] } (note the
    trailing space!), then the message will be received like so:

    \begin{verbatim}
    Subject: [My List] This is a message
    \end{verbatim}

    If you leave \code{subject_prefix} empty, no prefix will be added
    to the \mailheader{Subject}.  Mailman is careful not to add a
    prefix when the header already has one, as is the case in replies
    for example.  The prefix can also contain characters in the list's
    preferred language.  In this case, because of vagarities of the
    email standards, you may or may not want to add a trailing space.

\item[anonymous_list] --
    This variable allows you to turn on some simple anonymizing
    features of Mailman.  When you set this option to \emph{Yes},
    Mailman will remove or replace the \mailheader{From},
    \mailheader{Sender}, and \mailheader{Reply-To} fields of any
    message posted to the list.

    Note that this option is simply an aid for anonymization, it
    doesn't guarantee it.  For example, a poster's identity could be
    evident in their signature, or in other mail headers, or even in
    the style of the content of the message.  There's little Mailman
    can do about this kind of identity leakage.
\end{description}

\subsubsection{Reply-To header munging}

This section controls what happens to the \mailheader{Reply-To}
headers of messages posted through your list.

Beware!  \mailheader{Reply-To} munging is considered a religious issue
and the policies you set here can ignite some of the most heated
off-topic flame wars on your mailing lists.  We'll try to stay as
agnostic as possible, but our biases may still peak through.

\mailheader{Reply-To} is a header that is commonly used to redirect
replies to messages.  Exactly what happens when your uses reply to
such a message depends on the mail readers your users use, and what
functions they provide.  Usually, there is both a ``reply to sender''
button and a ``reply to all'' button.  If people use these buttons
correctly, you will probably never need to munge
\mailheader{Reply-To}, so the default values should be fine.

Since an informed decision is always best, here are links to two
articles that discuss the opposing viewpoints in great detail:

\begin{itemize}

\item \ulink{Reply-To Munging Considered
      Harmful}{http://www.unicom.com/pw/reply-to-harmful.html}
\item \ulink{Reply-To Munging Considered
      Useful}{http://www.metasystema.org/essays/reply-to-useful.mhtml}

\end{itemize}

The three options in this section work together to provide enough
flexibility to do whatever \mailheader{Reply-To} munging you might
(misguidingly :) feel you need to do.

\begin{description}

\item[first_strip_reply_to] --
    This variable controls whether any \mailheader{Reply-To} header
    already present in the posted message should get removed before
    any other munging occurs.  Stripping this header will be done
    regardless of whether or not Mailman will add its own
    \mailheader{Reply-To} header to the message.

    If this option is set to \emph{No}, then any existing
    \mailheader{Reply-To} header will be retained in the posted
    message.  If Mailman adds its own header, it will contain
    addresses which are the union of the original header and the
    Mailman added addresses.  The mail standards specify that a
    message may only have one \mailheader{Reply-To} header, but that
    that header may contain multiple addresses.

\item[reply_goes_to_list] --
    This variable controls whether Mailman will add its own
    \mailheader{Reply-To} header, and if so, what the value of that
    header will be (not counting original header stripping -- see
    above).

    When you set this variable to \emph{Poster}, no additional
    \mailheader{Reply-To} header will be added by Mailman.  This
    setting is strongly recommended.

    When you set this variable to \emph{This list}, a
    \mailheader{Reply-To} header pointing back to your list's posting
    address will be added.

    When you set this variable to \emph{Explicit address}, the value
    of the variable \code{reply_to_address} (see below) will be
    added.  Note that this is one situation where
    \mailheader{Reply-To} munging may have a legitimate purpose.  Say
    you have two lists at your site, an announce list and a discussion
    list.  The announce list might allow postings only from a small
    number of approved users; the general list membership probably
    can't post to this list.  But you want to allow comments on
    announcements to be posted to the general discussion list by any
    list member.  In this case, you can set the \mailheader{Reply-To}
    header for the announce list to point to the discussion list's
    posting address.

\item[reply_to_address] --
    This is the address that will be added in the
    \mailheader{Reply-To} header if \code{reply_goes_to_list} is set
    to \emph{Explicit address}.

\end{description}

\subsection{The Passwords Category}
\subsection{the Language Options Category}
\subsection{The Membership Management Category}
\subsection{The Non-digest Options Category}
\subsection{The Digest Options Category}
\subsection{The Privacy Options Category}
\subsection{The Bounce Processing Category}
\subsection{The Archiving Options Category}
\subsection{The Mail <-> News Gateway Category}
\subsection{The Auto-responder Category}
\subsection{The Content Filtering Category}
\subsection{The Topics Category}
\subsection{Emergency Moderation}

\section{Tending to Pending Moderator Requests}
\section{Editing the Public HTML Pages}
\section{Deleting the Mailing List}

\appendix

\section{This is an Appendix}

To create an appendix in a Python HOWTO document, use markup like
this:

\begin{verbatim}
\appendix

\section{This is an Appendix}

To create an appendix in a Python HOWTO document, ....


\section{This is another}

Just add another \section{}, but don't say \appendix again.
\end{verbatim}


\end{document}
